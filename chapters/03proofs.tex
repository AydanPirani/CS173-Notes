NOTE: The approaches defined here change based on universal and existential statements, and based on whether or not the proof is positive or negative. Be sure to take this into account!

\subsection{Proving a Universal Statement}
\begin{enumerate}
    \item Define all vocabulary (eg. rationals, integers, ...).
    \item Pick a representative value from the set (VARIABLE).
    \item Go from the value to the claim (hardest part to prove).
\end{enumerate}

\stepcounter{subsection}
\subsection{Direct Proof Outline}
\begin{itemize}
    \item Start with known information, move towards final statement.
    \item Sometimes need to reason backwards, but ALWAYS write forwards.
\end{itemize}

\subsection{Proving Existential Statements}
\begin{itemize}
    \item Find a value that matches the claim, done.
    \item Can choose any value, because the claim is existential.
\end{itemize}

\subsection{Disproving a Universal Statement}
\begin{itemize}
    \item Similar process to proving an existential statement.
    \item Find a value that proves it wrong.
\end{itemize}

\subsection{Disproving an Existential Statement}
\begin{itemize}
    \item Similar process to proving a universal statement.
    \item Find a representative element, then work from there.
\end{itemize}

\subsection{Recap of Proof Methods}
\begin{tabular}[!ht]{c|c|c}
    claim & prove & disprove\\ \hline
    universal & representative element & counterexample\\ 
    existential & example & representative element\\    
\end{tabular}

\stepcounter{subsection}
\stepcounter{subsection}

\subsection{Proof by Cases}
\begin{itemize}
    \item If claim is in the form p or q, then break it up, and prove for p and q individually.
    \item Combine both statements together afterwards.
\end{itemize}

\subsection{Rephrasing Claims}
\begin{itemize}
    \item Depending on the claim, might need to apply negations and DeMorgan's Laws to pull it all together.
\end{itemize}

\subsection{Proof by Contrapositive}
\begin{itemize}
    \item Fairly straightforward - if you can prove the contrapositive, then you have proven the claim.
    \item No strict rule as for when to use it.
\end{itemize}