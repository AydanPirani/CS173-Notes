\subsection{Trees}
\begin{itemize}
    \item Lots of applications, efficient way to store data.
\end{itemize}

\subsection{Defining Trees}
\begin{itemize}
    \item Has a root, and every node is connected to it.
    \item Node nearest the root is a \hl{parent}, below it is a \hl{child}.
    \item Two children of the same parent are \hl{siblings}.
    \item \hl{Leaf Node}: Node that has no children.
    \item \hl{Internal Node}: Node that has children.
    \item \hl{Levels}: How far a node is from the root.
    \item \hl{Height}: Maximum level of a node in the tree.
\end{itemize}

\subsection{m-ary Trees}
\begin{itemize}
    \item Each node can have up to m children.
    \item \hl{Full Tree}: Each internal node has m children.
    \item \hl{Complete Tree}: Each leaf node is at the same level.
\end{itemize}

\subsection{Height vs. Number of Nodes}
\begin{itemize}
    \item Two Cases: \begin{itemize}
        \item Full and Complete: $\log_2{n}$
        \item Not full and complete: between n - 1 and $\log_2{n}$
    \end{itemize}
\end{itemize}

\subsection{Context-Free Grammars}
\begin{itemize}
    \item \hl{Terminal Symbol}: Word or a character (but a way to end a string.)
    \item \hl{Parse Trees}:  structure of sequence of terminal symbols.
    \item \hl{Context-Free Grammar}: Set of rules that specify which children are possible. 
\end{itemize}