\subsection{Relations}
\begin{itemize}
    \item Set of ordered pairs of elements from A to elements in A.
    \item If (x, y) exists in the set, we denote it as xRy.
    \item Note that elements can relate to themselves.
    \item Can draw directed graphs to represent relations.
\end{itemize}

\subsection{Properties of Relations: Reflexive}
\begin{itemize}
    \item Three main cases: \begin{itemize}
        \item \hl{Reflexive} (every element relates to itself)
        \item \hl{Irreflexive} (no element relates to itself)
        \item Neither (some elements relate to themselves, but not all)
    \end{itemize}
\end{itemize}

\subsection{Symmetric and Antisymmetric}
\begin{itemize}
    \item If xRy and yRx, then relation is \hl{symmetric}.
    \item Represented in a graph by bidirectional arrows.
    \item \hl{Antisymmetric} if no two elements relate to each other in both directions.
\end{itemize}

\subsection{Transitive}
\begin{itemize}
    \item If aRb and bRc, then aRc.
    \item Whenever theres a n-length path in a graph from a to c, there must also be a length 1 path.
    \item NOTE: if aRb and bRa, then a must relate to itself in order for graph to be transitive.
\end{itemize}

\subsection{Types of Relations}
\begin{itemize}
    \item \hl{Partial Order}: Reflexive, antisymmetric, transitive.
    \item \hl{Linear Order}(total order): Partial order where all elements relate to each other.
    \item \hl{Strict Partial Order}: Irreflexive, antisymmetric, transitive.
    \item \hl{Equivalence Relation}: Reflexive, symmetric, transitive.
\end{itemize}