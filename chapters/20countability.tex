\subsection{The Rationals and the Reals}
\begin{itemize}
    \item Three sets of numbers: integers, rationals, reals.
    \item Integers are discrete whole numbers
    \item Majority of real numbers are irrational, and only a few are rational.
\end{itemize}

\subsection{Completeness}
\begin{itemize}
    \item Reals have completeness, rationals don't.
    \item Completeness: Any subset of reals with upper bound has a SMALLEST upper bound.
\end{itemize}

\subsection{Cardinality}
\begin{itemize}
    \item Two sets have the same cardinality iff there's a bijection from A to B.
    \item Eg. $f: \mathbb{R}\rightarrow\mathbb{Z}, f(n)\begin{cases}
         \frac{n}{2}, n\equiv0(mod2) \\
        \frac{-(n+1)}{2}, n\equiv1(mod2)
    \end{cases}$
    \item \hl{Countably Infinite}: Bijection exists from $\mathbb{N}$ or $\mathbb{Z}$ onto an infinite set A.
    \item \hl{Countable}: A set is countably infinite or finite. (Includes all subsets of integers).
\end{itemize}

\subsection{Cantor Schroeder Bernstein Theorem}
\begin{itemize}
    \item $|A| \leq |B|$ iff there's a one-to-one function from A to B.
    \item Can reverse this and do it twice (A to B and then B to A), and show that a bijection exists.
\end{itemize}