\begin{itemize}
    \item Most sets previously contained atomic elements (numbers, strings, tuples).
    \item Sets can also contain other sets.
    \item \hl{Collection}: A set that contains other sets.
\end{itemize}

\subsection{Sets Containing Sets}
\begin{itemize}
    \item Happens when we need to get subsets of another set.
    \item Can divide a set into an amount of (non)overlapping subsets.
    \item If a non-overlapping set of subsets is the domain of a function, each subset results in an an output. 
\end{itemize}

\subsubsection{Cardinality}
\begin{itemize}
    \item The cardinality of a set is the amount of elements in the set itself, not in its subsets.
    \item The empty set can be put in another set, and it counts as an element.
\end{itemize}

\subsection{Powersets and Set-Valued Functions}
\begin{itemize}
    \item Powerset of A contains ALL subsets of A (including empty set).
    \item Denote a powerset with $\mathbb{P}(A)$.
    \item Powerset of A contains $2^n$ elements, if A has n elements.
    \item Use powersets when a function returns MULTIPLE values, and hence consistency is important.
    \item Are useful for defining the domains of above-mentioned functions.
\end{itemize}

\subsection{Partitions}
\begin{itemize}
    \item Division of a base set A into non-overlapping subsets.
    \item Corresponds to equivalence reactions (and vice-versa).
    \item Three requirements for a partition of A:
    \begin{enumerate}
        \item Elements put together cover all of A.
        \item No empty set in partition.
        \item No overlap within elements of partition sets.
    \end{enumerate}
\end{itemize}

\subsection{Combinations}
\subsubsection{Combinations}
\begin{itemize}
    \item Combinations happen when we have an n-element set, from which we need all subsets of size k.
    \item \hl{K-Combination}: Subset of size k.
    \item Care about order in a permutation, but not in a combination.
\end{itemize}

\subsubsection{Equations}
\begin{itemize}
    \item Choose k elements without an order: $\frac{n!}{(n-k)!}$
    \item Choose k elements in order (AKA binomial coefficient): $C(n,k)={n \choose k}=\frac{n!}{k!(n-k)!}$
    \item Note that this is defined if $ n \geq k \geq 0 $
\end{itemize}

\subsection{Applying the Combinations Formula}
\begin{itemize}
    \item Used to select a set of locations and assign values.
    \item Might need to apply it multiple times to generate the correct answer.
\end{itemize}

\subsection{Combinations with Repetition}
\begin{itemize}
    \item Need a clever way to count the amount of possibilities with multiple groups.
    \item Replace each item with a star, and find amount of places you can put the separator.
    \item To choose k objects from a list of size n: ${{k + n - 1} \choose {n - 1}} = {{k + n - 1} \choose {k}}$ 
\end{itemize}

\subsection{Identities for Binomial Coefficients}
\begin{itemize}
    \item ${n \choose k}={n \choose {n-k}}$
    \item ${{n+1} \choose k} = {n \choose k} + {n \choose {k-1}}$
\end{itemize}